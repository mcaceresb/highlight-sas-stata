% ---------------------------------------------------------------------
% Program: test.tex
% Author:  github.com/mcaceresb
% Created: Sat May  7 19:23:06 EDT 2016
% Updated: Sun May  8 23:25:20 EDT 2016
% Purpose: File to test minted and listings output
% Note:    Run xelatex; change fonts if needed

%--------------------------------------------------------------
\documentclass[11pt]{article}

\usepackage{listings}
\usepackage{minted}
\usepackage{geometry}
\usepackage{color}

\input{../listings/lststata.tex}
% ---------------------------------------------------------------------
% Program: lststata.tex
% Author:  github.com/mcaceresb
% Created: Sat May  7 19:23:06 EDT 2016
% Updated: Sun May  8 22:25:39 EDT 2016
% Purpose: SAS language definition for LaTeX listings package

% Syntax from
% - syntax/sas.vim by James Kidd <james.kidd@covance.com>

\RequirePackage{listings}
\lstdefinelanguage{SAS}{
  sensitive=false,
  alsoletter={\%\&},
  %
  % User variables
  keywordsprefix=\&,
  %
  % Comments
  morecomment=[f][\color{Green}\slshape][0]*,
  morecomment=[s]{/*}{*/},
  %
  % Strings
  morestring=[b]",
  morestring=[d]',
  %
  % Datalines and cards
  morecomment=[s][\itshape\color{spYellow}]{datalines;}{;},
  morecomment=[s][\itshape\color{spYellow}]{cards;}{;},
  %
  % Distinct highlight for proc <proc>, data, run, quit
  morecomment=[s][\bfseries\color{DarkBlue}]{proc\ }{\ },
  morekeywords={},
  morekeywords=[2]{
    data ,proc ,run ,quit
  },
  %
  % Macros
  morekeywords=[3]{
    \%bquote ,\%nrbquote ,\%cmpres ,\%qcmpres ,\%compstor ,\%datatyp
    ,\%display ,\%do ,\%else ,\%end ,\%eval ,\%global ,\%goto ,\%if
    ,\%index ,\%input ,\%keydef ,\%label ,\%left ,\%length ,\%let
    ,\%local ,\%lowcase ,\%macro ,\%mend ,\%nrbquote ,\%nrquote
    ,\%nrstr ,\%put ,\%qcmpres ,\%qleft ,\%qlowcase ,\%qscan ,\%qsubstr
    ,\%qsysfunc ,\%qtrim ,\%quote ,\%qupcase ,\%scan ,\%str ,\%substr
    ,\%superq ,\%syscall ,\%sysevalf ,\%sysexec ,\%sysfunc ,\%sysget
    ,\%syslput ,\%sysprod ,\%sysrc ,\%sysrput ,\%then ,\%to ,\%trim
    ,\%unquote ,\%until ,\%upcase ,\%verify ,\%while ,\%window
  },
  %
  % Statements
  morekeywords=[4]{
    % conditionals
    do ,if ,then ,else ,end ,until ,while
    % SAS
    ,abort ,array ,attrib ,by ,call ,cards ,cards4 ,catname ,continue
    ,datalines ,datalines4 ,delete ,delim ,delimiter ,display ,dm ,drop
    ,endsas ,error ,file ,filename ,footnote ,format ,goto ,in ,infile
    ,informat ,input ,keep ,label ,leave ,length ,libname ,link ,list
    ,lostcard ,merge ,missing ,modify ,options ,output ,out ,page ,put
    ,redirect ,remove ,rename ,replace ,retain ,return ,select ,set
    ,skip ,startsas ,stop ,title ,update ,waitsas ,where ,window ,x
    ,systask
    % SQL
    ,add ,and ,alter ,as ,cascade ,check ,create ,delete ,describe
    ,distinct ,drop ,foreign ,from ,group ,having ,index ,insert ,into
    ,in ,key ,like ,message ,modify ,msgtype ,not ,null ,on ,or ,order
    ,primary ,references ,reset ,restrict ,select ,set ,table ,unique
    ,update ,validate ,view ,where
  },
  %
  % Built-in functions
  morekeywords=[5]{
    abs ,addr ,airy ,arcos ,arsin ,atan ,attrc ,attrn ,band ,betainv
    ,blshift ,bnot ,bor ,brshift ,bxor ,byte ,cdf ,ceil ,cexist ,cinv
    ,close ,cnonct ,collate ,compbl ,compound ,compress ,cos ,cosh ,css
    ,curobs ,cv ,daccdb ,daccdbsl ,daccsl ,daccsyd ,dacctab ,dairy
    ,date ,datejul ,datepart ,datetime ,day ,dclose ,depdb ,depdbsl
    ,depdbsl ,depsl ,depsl ,depsyd ,depsyd ,deptab ,deptab ,dequote
    ,dhms ,dif ,digamma ,dim ,dinfo ,dnum ,dopen ,doptname ,doptnum
    ,dread ,dropnote ,dsname ,erf ,erfc ,exist ,exp ,fappend ,fclose
    ,fcol ,fdelete ,fetch ,fetchobs ,fexist ,fget ,fileexist ,filename
    ,fileref ,finfo ,finv ,fipname ,fipnamel ,fipstate ,floor ,fnonct
    ,fnote ,fopen ,foptname ,foptnum ,fpoint ,fpos ,fput ,fread ,frewind
    ,frlen ,fsep ,fuzz ,fwrite ,gaminv ,gamma ,getoption ,getvarc
    ,getvarn ,hbound ,hms ,hosthelp ,hour ,ibessel ,index ,indexc
    ,indexw ,input ,inputc ,inputn ,int ,intck ,intnx ,intrr ,irr
    ,jbessel ,juldate ,kurtosis ,lag ,lbound ,left ,length ,lgamma
    ,libname ,libref ,log ,log10 ,log2 ,logpdf ,logpmf ,logsdf ,lowcase
    ,max ,mdy ,mean ,min ,minute ,mod ,month ,mopen ,mort ,n ,netpv
    ,nmiss ,normal ,note ,npv ,open ,ordinal ,pathname ,pdf ,peek ,peekc
    ,pmf ,point ,poisson ,poke ,probbeta ,probbnml ,probchi ,probf
    ,probgam ,probhypr ,probit ,probnegb ,probnorm ,probt ,put ,putc
    ,putn ,qtr ,quote ,ranbin ,rancau ,ranexp ,rangam ,range ,rank
    ,rannor ,ranpoi ,rantbl ,rantri ,ranuni ,repeat ,resolve ,reverse
    ,rewind ,right ,round ,saving ,scan ,sdf ,second ,sign ,sin ,sinh
    ,skewness ,soundex ,spedis ,sqrt ,std ,stderr ,stfips ,stname
    ,stnamel ,substr ,sum ,symget ,sysget ,sysmsg ,sysprod ,sysrc
    ,system ,tan ,tanh ,time ,timepart ,tinv ,tnonct ,today ,translate
    ,tranwrd ,trigamma ,trim ,trimn ,trunc ,uniform ,upcase ,uss ,var
    ,varfmt ,varinfmt ,varlabel ,varlen ,varname ,varnum ,varray
    ,varrayx ,vartype ,verify ,vformat ,vformatd ,vformatdx ,vformatn
    ,vformatnx ,vformatw ,vformatwx ,vformatx ,vinarray ,vinarrayx
    ,vinformat ,vinformatd ,vinformatdx ,vinformatn ,vinformatnx
    ,vinformatw ,vinformatwx ,vinformatx ,vlabel ,vlabelx ,vlength
    ,vlengthx ,vname ,vnamex ,vtype ,vtypex ,weekday ,year ,yyq ,zipfips
    ,zipname ,zipnamel ,zipstate
  },
  %
  % Literals
  morekeywords=[6]{
    null ,missing ,_all_ ,_automatic_ ,_character_ ,_n_ ,_infile_
    ,_name_ ,_null_ ,_numeric_ ,_user_ ,_webout_
  },
}

% ---------------------------------------------------------------------
% SAS enhanced editor style

\RequirePackage{color}
\RequirePackage[svgnames]{xcolor}
\definecolor{spYellow}{HTML}{ADAD00}
\providecommand{\textcolordummy}[2]{#2}

\lstalias{sas}{SAS}
\lstdefinestyle{sas-editor}{
    language          = SAS,
    showstringspaces  = false,   % Don't underline spaces in strings
    showspaces        = false,   % Don't underline spaces
    breaklines        = true,    % automatic line b\color{Blue}\color{Blue}reaking
    breakatwhitespace = true,    % breaks only at white space.
    %
    % User variables
    keywordstyle = {\bfseries\color{NavyBlue}\let\textcolor\textcolordummy},
    %
    % Distinct highlight for proc <proc>, data, run, quit
    keywordstyle = [2]{\bfseries\color{DarkBlue}},
    %
    % Built-in macro functions
    keywordstyle = [3]{\color{Blue}},
    %
    % Statements
    keywordstyle = [4]{\color{Blue}},
    %
    % Built-in functions
    keywordstyle = [5]{\color{Blue}},
    %
    % Built-in functions
    keywordstyle = [6]{\bfseries\color{Blue}},
    %
    % Strings and comments
    stringstyle  = \color{Purple},
    commentstyle = \color{Green}\slshape,
    %
    % Numbers Hack (I legit forget why I needed this)
    literate={0}{{\bfseries\textcolor{SeaGreen}{0}}}{1}%
             {1}{{\bfseries\textcolor{SeaGreen}{1}}}{1}%
             {2}{{\bfseries\textcolor{SeaGreen}{2}}}{1}%
             {3}{{\bfseries\textcolor{SeaGreen}{3}}}{1}%
             {4}{{\bfseries\textcolor{SeaGreen}{4}}}{1}%
             {5}{{\bfseries\textcolor{SeaGreen}{5}}}{1}%
             {6}{{\bfseries\textcolor{SeaGreen}{6}}}{1}%
             {7}{{\bfseries\textcolor{SeaGreen}{7}}}{1}%
             {8}{{\bfseries\textcolor{SeaGreen}{8}}}{1}%
             {9}{{\bfseries\textcolor{SeaGreen}{9}}}{1}%
             {.0}{{\bfseries\textcolor{SeaGreen}{.0}}}{2}%
             {.1}{{\bfseries\textcolor{SeaGreen}{.1}}}{2}%
             {.2}{{\bfseries\textcolor{SeaGreen}{.2}}}{2}%
             {.3}{{\bfseries\textcolor{SeaGreen}{.3}}}{2}%
             {.4}{{\bfseries\textcolor{SeaGreen}{.4}}}{2}%
             {.5}{{\bfseries\textcolor{SeaGreen}{.5}}}{2}%
             {.6}{{\bfseries\textcolor{SeaGreen}{.6}}}{2}%
             {.7}{{\bfseries\textcolor{SeaGreen}{.7}}}{2}%
             {.8}{{\bfseries\textcolor{SeaGreen}{.8}}}{2}%
             {.9}{{\bfseries\textcolor{SeaGreen}{.9}}}{2}%
}

% ---------------------------------------------------------------------
% Suggested settings

% \lstset{
%   basicstyle        = \setmonofont{DejaVu Sans Mono}\footnotesize\ttfamily,
%   tabsize           = 4,      % Tab size
%   showstringspaces  = false,  % Don't underline spaces in strings
%   showspaces        = false,  % Don't underline spaces
%   breaklines        = true,   % automatic line breaking
%   breakatwhitespace = true,   % breaks only at white space.
%   lineskip          = 1.5pt,  % Sparing between lines of code
%   commentstyle      = \color{black!50}\itshape \let\textcolor\textcolordummy,
% }

\lstset{
  basicstyle        = \setmonofont{DejaVu Sans Mono}\footnotesize\ttfamily,
  tabsize           = 4,      % Tab size
  showstringspaces  = false,  % Don't underline spaces in strings
  showspaces        = false,  % Don't underline spaces
  breaklines        = true,   % Automatic line breaking
  breakatwhitespace = true,   % Breaks only at white space.
  lineskip          = 1.5pt,  % Sparing between lines of code
  commentstyle      = \color{black!50}\itshape \let\textcolor\textcolordummy,
}

%--------------------------------------------------------------
% options

\geometry{
  margin = 1in,
  top = 0.75in,
  paperwidth  = 8.5in,
  paperheight = 11in,
}
\definecolor{bgray}{gray}{1}

% ---------------------------------------------------------------------
% fonts

\usepackage{fontspec}
% \fontspec{Open Sans}
% \setmainfont{Open Sans}
% \setmainfont{Open Sans Light}

\fontspec{Ubuntu}
\setmainfont{Ubuntu Light}
\setmonofont{Ubuntu Mono}
\newfontfamily\ubuntu{Ubuntu}

% ---------------------------------------------------------------------
\begin{document}

\begin{minted}[bgcolor=bgray,style=sas]{sas}
/**********************************************************************
 * Program: example.sas
 * Purpose: SAS Example for HighlightJS Plug-in
 **********************************************************************/

%put Started at %sysfunc(putn(%sysfunc(datetime()), datetime.));
options
    errors = 20  /* Maximum number of prints of repeat errors */
    fullstimer   /* Detailed timer after each step execution  */
;

%let maindir = /path/to/maindir;
%let outdir  = &maindir/out.;
systask command "mkdir -p &outdir." wait;
libname main "&maindir." access = readonly;

data testing;
    input name $ number delimiter = ",";
    datalines;
    John,1
    Mary,2
    Jane,3
    ;
    if number > 1 then final = 0;
    else do;
        final = 1;
    end;
run;

proc sql &sqlopts;
create table waffles as
    select * from testing;
quit;

%put NOTE: Hello;
%put NOTE- Hello;
%put WARNING: Hello;
%put ERROR: Hello;
%put Something ERROR- Hello;

%macro testMacro(positional, named = value);
    %put positional = &positional.;
    %put named      = log(&named.);
%mend testMacro;
%testMacro(positional, named = value);

dm 'clear log output odsresults';

proc datasets lib = work kill noprint; quit;
libname _all_ clear;

\end{minted}

\clearpage
\begin{lstlisting}[language=SAS,style=sas-editor]
/**********************************************************************
 * Program: example.sas
 * Purpose: SAS Example for HighlightJS Plug-in
 **********************************************************************/

%put Started at %sysfunc(putn(%sysfunc(datetime()), datetime.));
options
    errors = 20  /* Maximum number of prints of repeat errors */
    fullstimer   /* Detailed timer after each step execution  */
;

%let maindir = /path/to/maindir;
%let outdir  = &maindir/out.;
systask command "mkdir -p &outdir." wait;
libname main "&maindir." access = readonly;

data testing;
    input name $ number delimiter = ",";
    datalines;
    John,1
    Mary,2
    Jane,3
    ;
    if number > 1 then final = 0;
    else do;
        final = 1;
    end;
run;

proc sql &sqlopts;
create table waffles as
    select * from testing;
quit;

%put NOTE: Hello;
%put NOTE- Hello;
%put WARNING: Hello;
%put ERROR: Hello;

%macro testMacro(positional, named = value);
    %put positional = &positional.;
    %put named      = log(&named.);
%mend testMacro;
%testMacro(positional, named = value);

dm 'clear log output odsresults';

proc datasets lib = work kill noprint; quit;
libname _all_ clear;

\end{lstlisting}

\clearpage
\begin{minted}[bgcolor=bgray,style=stata]{stata}
program define excellentProgram
version 14.0

local hi  = `1'
local bye = `2'
local yes = ln(`hi')

* This is a comment
set obs `= _N + 1'
gen neg = 1 - 1 / (1 + exp(score))

/*
 * Multi line comments are pretty
 * because they span many lines
 */

reg y x
xi: reg y2 x i.dummy // This is another comment type

di "This is a normal string with a `local' $global ${global}"
di `"This is a "super string" that takes on anything"'
di "string`1'two${three}" bad `"string " "' good `"string " "'

// This also works at line starts
adopath ++ "${lib}/code/ado/"
cap adopath - SITE
cap adopath - PLUS
/*cap adopath - PERSONAL
cap adopath - OLDPLACE*/

forval i = 1 / 4{
  cap reg y x`i', robust
  if `i' == 2 {
    local c = _b[_cons]
    local b = _b[x`i']
    local x = ln(`i')
  }
}

* Something about how mata is really a second language within Stata
mata: mata mlib index
end
\end{minted}

\clearpage
\begin{lstlisting}[language=stata,style=stata-editor]
program define excellentProgram
version 14.0

local hi  = `1'
local bye = `2'
local yes = ln(`hi')

* This is a comment
set obs `= _N + 1'
gen neg = 1 - 1 / (1 + exp(score))

/*
 * Multi line comments are pretty
 * because they span many lines
 */

reg y x
xi: reg y2 x i.dummy // This is another comment type

di "This is a normal string with a `local' $global ${global}"
di `"This is a "super string" that takes on anything"'
di "string`1'two${three}" bad `"string " "' good `"string " "'

// This also works at line starts
adopath ++ "${lib}/code/ado/"
cap adopath - SITE
cap adopath - PLUS
/*cap adopath - PERSONAL
cap adopath - OLDPLACE*/

forval i = 1 / 4{
  cap reg y x`i', robust
  if `i' == 2 {
    local c = _b[_cons]
    local b = _b[x`i']
    local x = ln(`i')
  }
}

* Something about how mata is really a second language within Stata
mata: mata mlib index
end
\end{lstlisting}

% ---------------------------------------------------------------------
\end{document}
